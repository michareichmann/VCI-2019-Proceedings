\section{3D Pixel Detectors}
By placing column-like electrodes inside the detector material, the 3D geometry reduces the drift distance of a charge created by ionising particles compared to a planar device. More details about the working principle can be found in \cite{parker}, \cite{3D}. All devices discussed in this article were constructed with \ac{pCVD} diamond.
%%%%%%%%%%%%%%%%%%%%%%% WORKING PRINCIPLE %%%%%%%%%%%%%%%%%%%%%%%%%%%
% \subsection{Working Principle}
% Its basic principle is shown in \ar{3d1}: In a planar detector 
% \fig{.16}{3DConcept}[Comparison of the planar and the 3D detector concepts.][3d1]
% the readout and bias electrodes are brought onto the front and back side of the sensor with a thickness $D$. The resulting drift distance $L$ of the charge carriers is of the order of $D$. In the 3D detector the electrodes are put inside of the detector material so that the \acp{MIP} can travel the same distance $D$ in the material and therefore create the same amount charge carriers but $L$ is largely reduced.\par
%%%%%%%%%%%%%%%%%%%%%%% FABRICATION %%%%%%%%%%%%%%%%%%%%%%%%%%
\subsection{Fabrication}
In order to manufacture the electrodes in diamond, columns are 
fabricated using a \SI{130}{\femto\second} laser with a wavelength of \SI{800}{\nano\meter} which is used to convert the diamond into a electrically resistive mixture of different carbon phases \cite{3dfab}. A \ac{SLM} was used to correct aberrations during fabrication to achieve a column yield of \SI{>99}{\%}, a column diameter of \SI{2.6}{\micro\meter} and a resistivity of the columns of the order of \SIrange{.1}{1}{\ohm\centi\meter} \cite{slm}. The largest fabricated device had about 4000 3D cells, where one cell consists of four bias electrodes and one readout electrode in the centre. \par
The detector is constructed by connecting to the bias and readout columns with surface metallisation and bump bonding the sensor to the readout electronics as shown in \ar{bb}. For the detectors described herein a cell size of \SI{50x50}{\micro\meter} was chosen. Since the layout of the available \acp{ROC} has a different pixel pitch several cells were ganged together.
\fig{.10}{BondingScheme}[Bump bonding scheme.][bb]
%%%%%%%%%%%%%%%%%%%%%%% 3x2 %%%%%%%%%%%%%%%%%%%%%%%%%%
\subsection{PSI46digV2.1respin readout}
The first prototype of a \SI{50x50}{\micro\meter} 3D pixel detector was connected to the PSI46digV2.1respin \ac{ROC} \cite{kornmayer} with a \SI{3x2}{} cell ganging to match the pixel pitch of \SI{150x100}{\micro\meter}. The 3D sensors were bump bonded to the \ac{ROC} at the Nanofabrication Lab at the Princeton University with indium bumps by putting equal height indium columns on both \ac{ROC} and the sensor and then pressing them together.\par
The hit efficiency is defined as the percentage of hits in the 3D pixel detector when a particle track traversed the detector. The preliminary beam test results show that, relative to a planar silicon device, the efficiency in the fiducial area was \SI{99.3}{\%} (\ar{em2}). This efficiency estimation does not account for non-working 3D cells in this region which can happen due to broken or missing columns or due to metalisation issues. In order to acquire this information further data should be analysed. Nevertheless, a small mismatch between a 3D and a planar device is expected due to regions inside of the detector where the electric field is low \cite{guilio} and the relatively inefficient columns themselves. \ar{ev} shows that the device plateaus at a voltage of \SI{30}{\volt}. The preliminary analysis of the pulse height distribution yields a mean value of \SI{\sim11}{\kilo e}. The precise pulse height calibration of the \ac{ROC} is currently being studied.
\subfigs{\subfig{.16}{EffMap}[Efficiency map. The red box marks the fiducial area.][em2]}{\subfig{.16}{EffVol}[Efficiency vs. voltage in the fiducial area.][ev]}[Hit efficiency results with PSI46digV2.1respin readout.]
%%%%%%%%%%%%%%%%%%%%%%% 1x5 %%%%%%%%%%%%%%%%%%%%%%%%%%
\subsection{FE-I4b readout}
The second prototype was connected to the FE-I4b \ac{ROC} \cite{fei4} with a \SI{5x1}{} cell ganging due to the \ac{ROC} pitch of \SI{250x50}{\micro\meter}. The bump bonding was performed at IFAE-CNM in Barcelona by an adapted process with tin-silver bumps. Using a high resolution beam telescope with a spatial resolution of \SI{3}{\micro\meter} at the device under test the efficiency could be mapped to the spatial coordinates. The analysis yields an efficiency of \SI{97.8}{\%} in the contiguous fiducial area (\ar{em1}).
\vspace*{-10pt}\fig{.195}{EffAtL1}[Hit efficiency results with the FE-I4b readout. The red box denotes the fiducial area.][em1]
The lower than \SI{99}{\%} efficiency is most likely due to issues with the bump bonding or the metallisation. The preliminary pulse height in the fiducial region was \SI{\sim15}{\kilo e} which is consistent with the result of the first prototype considering the different momenta of the incident particles. The precise pulse height calibration for the FE-I4b \ac{ROC} is in the process of being performed.
