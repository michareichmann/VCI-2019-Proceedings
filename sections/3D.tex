\section{3D Pixel Detectors}
By placing the column like electrodes inside the detector material, the 3D geometry reduces the drift distance of the charge carriers created by ionising particles compared to a planar device. More details about the working principle can be found in \cite{parker}, \cite{3D}. All devices discussed in the following were build on \ac{pCVD} diamond.
%%%%%%%%%%%%%%%%%%%%%%% WORKING PRINCIPLE %%%%%%%%%%%%%%%%%%%%%%%%%%%
% \subsection{Working Principle}
% Its basic principle is shown in \ar{3d1}: In a planar detector 
% \fig{.16}{3DConcept}[Comparison of the planar and the 3D detector concepts.][3d1]
% the readout and bias electrodes are brought onto the front and back side of the sensor with a thickness $D$. The resulting drift distance $L$ of the charge carriers is of the order of $D$. In the 3D detector the electrodes are put inside of the detector material so that the \acp{MIP} can travel the same distance $D$ in the material and therefore create the same amount charge carriers but $L$ is largely reduced.\par
%%%%%%%%%%%%%%%%%%%%%%% FABRICATION %%%%%%%%%%%%%%%%%%%%%%%%%%
\subsection{Fabrication}
In order to generate the electrodes in diamond, columns are 
\wrapfig[l]{.5}{BondingScheme}[Bump bonding scheme.][bb]
drilled using a \SI{130}{\femto\second} laser with a wavelength of \SI{800}{\nano\meter} which converts the diamond into a electrically conductive mixture of different carbon phases \cite{3dfab}. The conductivity of the columns is of the \orderof{\SI{1}{(\ohm\centi\meter)^{-1}}}. By applying \ac{SLM} a column yield of \SI{>99}{\%} and a column diameter of \SI{2.6}{\micro\meter} can be achieved \cite{slm}. The largest fabricated device has about 4000 3D cells, where one cell consists out of four bias electrodes and one read-out electrode in the centre. \par
The detector is built by connecting to the bias and readout columns with surface metallisation and bump bonding the sensor to the readout electronics as shown in \ar{bb}. For the detectors described in here a cell size of \SI{50x50}{\micro\meter} was chosen. Since the layout of the available \acp{ROC} has a different pixel pitch several cells had to be ganged together.
%%%%%%%%%%%%%%%%%%%%%%% 3x2 %%%%%%%%%%%%%%%%%%%%%%%%%%
\subsection{PSI46digV2.1respin read-out}
The first prototype of a \SI{50x50}{\micro\meter} 3D pixel detector was connected to the PSI46digV2.1respin \ac{ROC} \cite{kornmayer} with a \SI{3x2}{} cell ganging to match the pixel pitch of \SI{150x100}{\micro\meter}. The 3D sensors were bump bonded to the \ac{ROC} at the Nanofabrication Lab in Princeton with indium bumps by putting equal height indium columns on both \ac{ROC} and the sensor and then pressing them together. The preliminary beam test results show that relative to a planar silicon device the efficiency in the fiducial area amounts to \SI{99.3}{\%} (\ar{em2}). This efficiency estimation does not account for non-working 3D cells in this region which can happen due to broken or missing columns or due to metalisation issues. In order to acquire this information further data has to be analysed. Nevertheless, a small mismatch between a 3D and a planar device is expected due to regions inside of the detector where the electric field is low \cite{guilio} and the columns themselves. \ar{ev} shows that the device already plateaus at a voltage of \SI{30}{\volt}. The preliminary analysis of the pulse height distribution 
\subfigs{\subfig{.16}{EffMap}[\SI{3x2}{} efficiency. The red box marks the fiducial area.][em2]}{\subfig[.45][-.02]{.16}{EffVol}[Efficiency vs. voltage.][ev]}[\SI{3x2}{} results.]
yields a mean value of \SI{\sim11}{\kilo e}. The updated pulse height information will be reported after the precise pulse height calibration of the \ac{ROC} is performed.
%%%%%%%%%%%%%%%%%%%%%%% 1x5 %%%%%%%%%%%%%%%%%%%%%%%%%%
\subsection{FE-I4b read-out}
The second prototype was connected to the FE-I4b 
\fig{.25}{EffAtL1}[\SI{5x1}{} efficiency. The red box denotes the fiducial area.][em1]
\ac{ROC} \cite{fei4} with a \SI{5x1}{} cell ganging due to its pitch of \SI{250x50}{\micro\meter}. The bump bonding was performed at IFAE-CNM in Barcelona by an adapted process with tin-silver bumps. Using a high resolution beam telescope with a spatial resolution of \SI{3}{\micro\meter} at the device under test the efficiency could be mapped to the spatial coordinates. The results yield an efficiency of \SI{97.8}{\%} in the contiguous fiducial area (\ar{em1}). The low efficiency is most likely due to issues with the bump bonding or the metallisation. The preliminary pulse height in the fiducial region amounts to \SI{\sim15}{\kilo e} which is consistent with the result of the first prototype considering the different momenta of the incident particles. An updated pulse height will be reported as soon as the precise pulse height calibration for this \ac{ROC} is performed.
