\section{3D Pixel Detectors}
3D detectors can largely reduce the drift distance of the charge carriers induced by ionising particles. More details about the working principle can be found in \cite{3D}, \cite{parker}.\par
%%%%%%%%%%%%%%%%%%%%%%% WORKING PRINCIPLE %%%%%%%%%%%%%%%%%%%%%%%%%%%
% \paragraph{Working Principle}
% Its basic principle is shown in \ar{3d1}: In a planar detector 
% \fig{.16}{3DConcept}[Comparison of the planar and the 3D detector concepts.][3d1]
% the readout and bias electrodes are brought onto the front and back side of the sensor with a thickness $D$. The resulting drift distance $L$ of the charge carriers is of the order of $D$. In the 3D detector the electrodes are put inside of the detector material so that the \acp{MIP} can travel the same distance $D$ in the material and therefore create the same amount charge carriers but $L$ is largely reduced.\par
%%%%%%%%%%%%%%%%%%%%%%% FABRICATION %%%%%%%%%%%%%%%%%%%%%%%%%%
\paragraph{Fabrication}
In order to generate these electrodes in diamond the columns are 
\wrapfig[r]{.35}{BondingScheme}[Bump bonding scheme.][bb]
drilled with a \SI{800}{\nano\meter} femtosecond laser which converts the diamond into a resistive mixture of different carbon phases \cite{3dfab}. By using \ac{SLM} a column yield of \SI{>99}{\%} and a column diameter of \SI{2.6}{\micro\meter} can be achieved \cite{slm}. The largest fabricated device has about 4000 of these cells. \par
The final detector is then built by connecting to the bias and readout columns with surface metallisation and bump bonding the sensor to the readout electronics as shown in \ar{bb}. For the latest detectors a cell size of \SI{50x50}{\micro\meter} was chosen. Since the layout of the available \acp{ROC} has a different pixel pitch several cells had to be ganged together.\par
%%%%%%%%%%%%%%%%%%%%%%% 3x2 %%%%%%%%%%%%%%%%%%%%%%%%%%
\paragraph{PSI46digV2.1respin}
The first prototype of a \SI{50x50}{\micro\meter} 3D pixel detector is connected to the CMS
\subfigs{\subfig{.2}{EffMap}[\SI{3x2}{} efficiency.][em2]}{\subfig{.2}[r]{EffVol}[Efficiency vs. voltage.][ev]}[\SI{3x2}{} results.]
PSI46digV2.1respin \ac{ROC} \cite{kornmayer} with a \SI{3x2}{} ganging to match the pixel pitch of \SI{150x100}{\micro\meter}. In this case indium bump are used to connect to the \ac{ROC}. %The chip was then tuned to a global pixel threshold of \SI{\sim1500}{e}. 
The preliminary beam test results show an efficiency of \SI{99.2}{\%} (\ar{em2}). This value is close to the efficiency of a silicon pixel of \SI{99.9}{\%} which was tested in parallel. Compared to the silicon the 3D pixel detector has a relative efficiency of \SI{99.3}{\%}. For this value it was not considered that some of the 3D cells are not working. Due to the bad tracking resolution at \ac{PSI} it is not possible to detect them in this measurement. Nevertheless a small mismatch is expected due to low field regions between the electrodes and the columns themselves. \ar{ev} shows the the device is already fully efficient at a Voltage of \SI{30}{\volt} and stays constant for higher voltages.\par
The measured pulse height of \SI{\sim11}{\kilo e} is consistent with the measurement in the next paragraph due to the different momentum of the incident particles. 

%%%%%%%%%%%%%%%%%%%%%%% 1x5 %%%%%%%%%%%%%%%%%%%%%%%%%%
\paragraph{FE-I4}
The second prototype is connected to the ATLAS FE-I4 \ac{ROC} \cite{fei4}
with a \SI{1x5}{} ganging due to its pitch of \SI{50x250}{\micro\meter} . The bump bonding was done using a adapted process with tin-silver bumps. Using a beam telescope with \SI{3}{\micro\meter} spatial resolution the efficiency could could be mapped to the spatial coordinates yielding an efficiency of \SI{97.8}{\%} over a contiguous area (\ar{em1}). However there are many small regions where the device is almost completely inefficient. 
\wrapfig[r]{.4}{EffAtL1}[\SI{1x5}{} efficiency][em1]
The reason for this are most likely issues with the bump bonding or the metallisation. Nevertheless the preliminary pulse height in the good region already amount to \SI{\sim15}{\kilo e} which corresponds to a \ac{CCE} of more than \SI{80}{\%}. 
%TODO pulse height
