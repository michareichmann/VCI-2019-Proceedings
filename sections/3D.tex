\section{3D Pixel Detectors}

The radiation damage created by the \ac{HL-LHC} will become a big challenge for the innermost tracking detectors. 
After large irradiation, all detector materials become trap limited with a \ac{MDD} below \SI{75}{\micro\meter}. The concept of a 3D detector is a possible way to reduce the drift distance below the this \ac{MDD}. More details about the fabrication and the functionality can be found in \cite{3D}, \cite{parker}.\par
%%%%%%%%%%%%%%%%%%%%%%% WORKING PRINCIPLE %%%%%%%%%%%%%%%%%%%%%%%%%%%
\paragraph{Working Principle}
Its basic principle is shown in \ar{3d1}: In a planar detector 
\fig{.16}{3DConcept}[Comparison of the planar and the 3D detector concepts.][3d1]
the readout and bias electrodes are brought onto the front and back side of the sensor with a thickness $D$. The resulting drift distance $L$ of the charge carriers is of the order of $D$. In the 3D detector the electrodes are put inside of the detector material so that the \acp{MIP} can travel the same distance $D$ in the material and therefore create the same amount charge carriers but $L$ is largely reduced.\par
%%%%%%%%%%%%%%%%%%%%%%% FABRICATION %%%%%%%%%%%%%%%%%%%%%%%%%%
\paragraph{Fabrication}
In order to generate these electrode in diamond the columns are 
\wrapfig[r]{.35}{BondingScheme}[Bump bonding scheme.][bb]
drilled with a \SI{800}{\nano\meter} femtosecond laser which converts the diamond into a resistive mixture of different carbon phases \cite{3dfab}. By using \ac{SLM} a column yield of \SI{>99}{\%} and a column diameter of \SI{2.6}{\micro\meter} can be achieved \cite{slm}.\par
The final detector is then built by connecting to the bias and readout columns with surface metallisation and bump bonding the sensor to the readout electronics as shown in \ar{bb}. For the latest detectors a cell size of \SI{50x50}{\micro\meter} was chosen, but since the layout of the available chips had a different pixel pitch several cells had to be ganged together.\par
%%%%%%%%%%%%%%%%%%%%%%% 1x5 %%%%%%%%%%%%%%%%%%%%%%%%%%
\paragraph{\SI{1x5}{} Ganging}
In order to connect to the ATLAS FEI4 \ac{ROC} 
\wrapfig[r]{.35}{EffAtL}[\SI{1x5}{} efficiency][em1]
with a pitch of \SI{50x250}{\micro\meter} a \SI{1x5}{} ganging was required. The bump bonding was done using a adapted process with tin-silver bumps. Using a beam telescope with \SI{3}{\micro\meter} spatial resolution the efficiency could could be mapped to the spatial coordinates yielding an efficiency of \SI{97.8}{\%} over a connected area.
%TODO pulse height
%%%%%%%%%%%%%%%%%%%%%%% 3x2 %%%%%%%%%%%%%%%%%%%%%%%%%%
\paragraph{\SI{3x2}{} Ganging}
Indium bump bonds and a \SI{3x2}{} ganging are used to connect 
\subfigs{\subfig{.2}{EffMap}[\SI{3x2}{} efficiency.][em2]}{\subfig{.2}{EffVol}[Efficiency vs. voltage.][ev]}
to the CMS PSI46digV2.1respin \ac{ROC} \cite{kornmayer}. The chip was then tuned to a global pixel threshold of \SI{\sim1500}{e}. The preliminary beam test results show an efficiency of \SI{99.2}{\%}. This value is close to the efficiency of a silicon pixel of \SI{99.9}{\%} which was tested in parallel. Compared to the silicon the 3D pixel detector has a relative efficiency of \SI{99.3}{\%}. The loss of \SI{0.6}{\%} is believed to originate from the low field regions between the electrodes.
