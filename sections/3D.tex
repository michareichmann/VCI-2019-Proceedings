\section{3D Pixel Detectors}
The 3D geometry reduces the drift distance of the charge carriers created by ionising particles compared to a planar device. More details about the working principle can be found in \cite{parker}, \cite{3D}.
%%%%%%%%%%%%%%%%%%%%%%% WORKING PRINCIPLE %%%%%%%%%%%%%%%%%%%%%%%%%%%
% \subsection{Working Principle}
% Its basic principle is shown in \ar{3d1}: In a planar detector 
% \fig{.16}{3DConcept}[Comparison of the planar and the 3D detector concepts.][3d1]
% the readout and bias electrodes are brought onto the front and back side of the sensor with a thickness $D$. The resulting drift distance $L$ of the charge carriers is of the order of $D$. In the 3D detector the electrodes are put inside of the detector material so that the \acp{MIP} can travel the same distance $D$ in the material and therefore create the same amount charge carriers but $L$ is largely reduced.\par
%%%%%%%%%%%%%%%%%%%%%%% FABRICATION %%%%%%%%%%%%%%%%%%%%%%%%%%
\subsection{Fabrication}
In order to generate the electrodes in diamond 
\wrapfig[l]{.5}{BondingScheme}[Bump bonding scheme.][bb]
columns are drilled using a \SI{130}{\femto\second} laser with a wavelength of \SI{800}{\nano\meter} which converts the diamond into a electrically conductive mixture of different carbon phases \cite{3dfab}. The conductivity of the columns is of the \orderof{\SI{1}{\ohm\centi\meter}}. By applying \ac{SLM} a column yield of \SI{>99}{\%} and a column diameter of \SI{2.6}{\micro\meter} can be achieved \cite{slm}. The largest fabricated device has about 4000 of these cells. \par
The final device was built by connecting to the bias and readout columns with surface metallisation and bump bonding the sensor to the readout electronics as shown in \ar{bb}. For the detectors described in here a cell size of \SI{50x50}{\micro\meter} was chosen. Since the layout of the available \acp{ROC} has a different pixel pitch several cells had to be ganged together.
%%%%%%%%%%%%%%%%%%%%%%% 3x2 %%%%%%%%%%%%%%%%%%%%%%%%%%
\subsection{PSI46digV2.1respin read-out}
The first prototype of a \SI{50x50}{\micro\meter} 3D pixel detector was connected to the PSI46digV2.1respin \ac{ROC} \cite{kornmayer} with a \SI{3x2}{} cell ganging to match the pixel pitch of \SI{150x100}{\micro\meter}. In this case the 3D sensors were bump bonded to the \ac{ROC} at the Nanofabrication Lab in Princeton with indium bumps. The preliminary beam test results show, that relative to a planar silicon device the efficiency in the 
\subfigs{\subfig{.16}{EffMap}[\SI{3x2}{} efficiency.][em2]}{\subfig{.16}[r]{EffVol}[Efficiency vs. voltage.][ev]}[\SI{3x2}{} results.]
red fiducial area amounts to \SI{99.3}{\%} (\ar{em2}). It was not considered that some of the 3D cells in this region may not work due to broken or missing columns or due to metalisation issues. In order to acquire this information further data has to be analysed. Nevertheless a small mismatch is expected due to low field regions between the electrodes and the columns themselves. \ar{ev} shows the device is already fully efficient at a Voltage of \SI{30}{\volt}. The preliminary pulse height distribution yields a mean value of \SI{\sim11}{\kilo e}. Since the calibration of the \ac{ROC} has not been completed the exact value has yet to be determined.
%%%%%%%%%%%%%%%%%%%%%%% 1x5 %%%%%%%%%%%%%%%%%%%%%%%%%%
\subsection{FE-I4b read-out}
The second prototype was connected to the FE-I4b 
\fig{.25}{EffAtL1}[\SI{5x1}{} efficiency][em1]
\ac{ROC} \cite{fei4} with a \SI{5x1}{} cell ganging due to its pitch of \SI{250x50}{\micro\meter}. The bump bonding was performed at IFAE-CNM in Barcelona by an adapted process with tin-silver bumps. Using a high resolution beam telescope with a spatial resolution of \SI{3}{\micro\meter} at the device under test the efficiency could be mapped to the spatial coordinates. The results yield an efficiency of \SI{97.8}{\%} over the contiguous area in the red box (\ar{em1}). The missing efficiency is most likely due to issues with the bump bonding or the metallisation. Nevertheless the preliminary pulse height in the fiducial region already amounts to \SI{\sim15}{\kilo e} which is consistent with the result of the first prototype considering the different momenta of the incident particles. A precise calibration for this \ac{ROC} is also under way.
