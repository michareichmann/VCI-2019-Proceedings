\begin{abstract}
The RD42 collaboration is investigating \ac{CVD} diamond as a future material for particle detectors in a harsh radiation environment. The latest beam test results of 3D pixel detectors fabricated with \ac{pCVD} diamonds are discussed. The cells of the devices have a size of \SI{50x50}{\micro\meter} with columns \SI{2.6}{\micro\meter} in diameter. The cells were ganged in a \SI{3x2}{} and a \SI{5x1}{} pattern to match the layouts of the pixel read-out chips currently used in the CMS and ATLAS experiments at the LHC, respectively. In beam tests, using tracks reconstructed with a high precision tracking telescope, both devices achieved tracking efficiencies greater than \SI{97}{\%}. The efficiency of both devices plateaus at a bias voltages of \SI{30}{\volt}. The latest high rate beam test results of irradiated \ac{pCVD} diamond pad detectors are also presented. In many measurements with with particle fluxes up to \SI{20}{\mhzcm} and irradiations up to \SI{8e15}{\ncm} it could be shown that irradiated \ac{pCVD} diamonds do not depend on flux to the \orderof{\SI{2}{\%}}.
\end{abstract}
