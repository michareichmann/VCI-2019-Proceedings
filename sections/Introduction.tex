\section{Introduction}
%%%%%%%%%%%%%%%%%%%%%%% MOTIVATION %%%%%%%%%%%%%%%%%%%%%%%%%%
The radiation levels of the \ac{HL-LHC} are expected to be a big challenge for the future detectors. By 2028 experiments must be prepared for an instantaneous luminosity of \SI{7.5e34}{\per\centi\meter\squared\per\second}. In this environment the innermost tracking layer at a transverse distance of \SI{\sim30}{\milli\meter} to the \acl{IP} will be exposed to a total fluence of \SI{2e16}{n_{eq}\per\centi\meter^2} which corresponds to a total dose of the \orderof{\SI{10}{\mega\gray}} \cite{dose}. After such a large dose, all detector materials become trap limited with a schubweg well below \SI{75}{\micro\meter}.  The expected lifetime of the current planar silicon tracking detectors would be about one year in such and environment.\par
Due to the properties of \ac{CVD} diamond, such as the displacement energy of \SI{42}{\electronvolt\per atom} and the band gap of \SI{5.5}{\electronvolt}, the RD42 collaboration is investigating it as a possible detector material \cite{rd42}. Compared to analogous silicon detectors,  various studies have shown that diamond is at a minimum three times more radiation hard \cite{deboer}, collects the charges at least two times faster \cite{pernegger} and conducts heat four times more efficiently \cite{zhao}.\par
%%%%%%%%%%%%%%%%%%%%%%% INTRODUCTION %%%%%%%%%%%%%%%%%%%%%%%%%%
By now the technology of diamond detectors is well established in high energy physics. Many high energy physics experiments are already using \aclp{BCM} or \aclp{BLM} based on \ac{CVD} diamonds \cite{babar, bcm, dbm1}.\par
The RD42 collaboration is studying a novel detector design in diamond, namely 3D detectors. The 3D concept reduces the drift distance an electron-hole pair must undergo to reach an electrode below the schubweg of an irradiated sensor without reducing the amount of created electron-hole pairs.\par
The particle flux of the \ac{HL-LHC} will also reach a completely new regime. Hence it is important to study all proposed detectors at high rates of particles.
