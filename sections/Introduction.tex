\section{Introduction}
%%%%%%%%%%%%%%%%%%%%%%% MOTIVATION %%%%%%%%%%%%%%%%%%%%%%%%%%
\paragraph{Motivation} 
The radiation damage created by the \ac{HL-LHC} will become a big challenge for the detector. By 2028 an instantaneous luminosity of \SI{5e34}{\per\centi\meter\squared\per\second} is expected \cite{hllhc}. In this environment the innermost tracking layer at a distance of \SI{\sim30}{\milli\meter} to the \ac{IP} is expected to be exposed to a total fluence of \SI{2e16}{n_{eq}\per \centi\meter^2} by this time \cite{auzinger}.\par
After large irradiation, all detector materials become trap limited with a \ac{MDD} below \SI{75}{\micro\meter}. Due to its great properties, such as the large displacement energy of \SI{42}{\electronvolt\per atom} and the high band gap of \SI{5.5}{\electronvolt}, the RD42 collaboration is investigating \ac{CVD} diamond possible candidate. In various studies it was shown that compared to corresponding silicon detectors, diamond is at minimum three times more radiation hard \cite{deboer}, has at least a two times faster charge collection \cite{pernegger} and its thermal conductivity is four times higher \cite{zhao}.\par
%%%%%%%%%%%%%%%%%%%%%%% INTRODUCTION %%%%%%%%%%%%%%%%%%%%%%%%%%
\paragraph{Introduction}
By now the technology of diamond detectors is well established in high energy physics. Many of the experiments are already using \acp{BCM} or \acp{BLM} based on \ac{CVD} diamonds.\par
The RD42 collaboration is also investigating a novel detector design, namely 3D detectors. This concept is a possible way to reduce the drift distance below a critical \ac{MDD} without reducing the total number of the created electron-hole pairs. Since the particle flux of the \ac{HL-LHC} will be in completely new regime, high rate studies of pad detectors are performed at \ac{PSI} with nearly \acp{MIP} and tunable particle fluxes up to \orderof{\SI{10}{\mhzcm}}.
