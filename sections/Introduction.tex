\section{Introduction}
%%%%%%%%%%%%%%%%%%%%%%% MOTIVATION %%%%%%%%%%%%%%%%%%%%%%%%%%
The radiation levels of the \ac{HL-LHC} will become a big challenge for the future detectors. By 2028 an instantaneous luminosity of \SI{5e34}{\per\centi\meter\squared\per\second} is expected \cite{hllhc}. In this environment the innermost tracking layer at a distance of \SI{\sim30}{\milli\meter} to the \acl{IP} is expected to be exposed to a total fluence of \SI{2e16}{n_{eq}\per \centi\meter^2} \cite{auzinger}.\par
After large fluence, all detector materials become trap limited with a \acl{MDD} below \SI{75}{\micro\meter}. Due to its properties, such as the displacement energy of \SI{42}{\electronvolt\per atom} and the band gap of \SI{5.5}{\electronvolt}, the RD42 collaboration is investigating \ac{CVD} diamond possible detector material \cite{rd42}. In various studies it was shown that compared to corresponding silicon detectors, diamond is at a minimum three times more radiation hard \cite{deboer}, collecting the charges at least two times faster \cite{pernegger} and conducting heat four times more efficiently \cite{zhao}.\par
%%%%%%%%%%%%%%%%%%%%%%% INTRODUCTION %%%%%%%%%%%%%%%%%%%%%%%%%%
By now the technology of diamond detectors is well established in high energy physics. Many high energy physics experiments are already using \aclp{BCM} or \aclp{BLM} based on \ac{CVD} diamonds \cite{babar}, \cite{bcm}, \cite{dbm1}.\par
The RD42 collaboration is also investigating a novel detector design, namely 3D detectors. This concept is a possible way to reduce the drift distance below the \acl{MDD} of an irradiated sensor without reducing the total number of the excited electron-hole pairs. Since the particle flux of the \ac{HL-LHC} will be in completely new regime, high rate studies of pad detectors are performed at \ac{PSI} with nearly \acp{MIP} and tunable particle fluxes up to \orderof{\SI{10}{\mhzcm}}.
