\section{Conclusion}
There is great progress in the development of more radiation tolerant particle detectors based on \ac{pCVD} diamonds. The working principle of 3D pixel detectors was proven with great success down to cell sizes of \SI{50x50}{\micro\meter} and column diameters of \SI{2.6}{\micro\meter}. The number of cells could be scaled up by a factor of 40 to 4000. For the first time more than \SI{80}{\%} of the created charge in the material was read out. The efficiency of the column drilling process is now above \SI{99}{\%} and the relative efficiency of the 3D pixel detectors is \SI{99.3}{\%} compared to a silicon device.\par
In extensive studies it was found that irradiated \ac{pCVD} diamond detectors work reliably and show no signal dependence to the \orderof{\SI{2}{\%}} up to an incident particle flux of \SI{20}{\mhzcm}. This was shown for an irradiation  up to a fluence of \SI{8e15}{\ncm}. The leakage current of a good detector is proportional to the flux and of the \orderof{\SI{1}{\nano\ampere}} without beam. Some diamonds show a strong rate dependence which is most likely to surface contamination and is possible to overcome.
