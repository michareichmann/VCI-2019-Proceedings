\section{Conclusion}
There is progress in the development of radiation tolerant particle detectors based on \ac{pCVD} diamonds. The working principle of 3D pixel detectors was proven down to cell sizes of \SI{50x50}{\micro\meter} and column diameters of \SI{2.6}{\micro\meter}. The largest device has a number of \SI{4000}{cells}. Already the first prototypes of small cell 3D detectors read out more charge than any planar \ac{pCVD} diamond detector. The efficiency of the column drilling process is now above \SI{99}{\%}. The measured relative hit efficiency of the 3D pixel detectors reaches \SI{99.3}{\%} compared to a silicon device.\par
In extensive studies it was found that irradiated \ac{pCVD} diamond detectors work reliably and show no signal dependence to the \orderof{\SI{2}{\%}} up to an incident particle flux of \SI{20}{\mhzcm}. This was shown for an irradiation up to a fluence of \SI{8e15}{\ncm}. The beam induced current of a \ac{pCVD} diamond is proportional to the flux and the leakage current is of the \orderof{\SI{1}{\nano\ampere}}. A small fraction of the diamonds shows a large rate dependence which is most likely to surface issues and is possible to correct.
