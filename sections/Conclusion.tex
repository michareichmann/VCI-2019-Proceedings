\section{Conclusion}
There is progress in the development of radiation tolerant particle detectors based on \ac{pCVD} diamonds. The working principle of 3D diamond pixel detectors was proven for cell sizes of \SI{50x50}{\micro\meter} and column diameters of \SI{2.6}{\micro\meter}. The largest device had a number of \SI{4000}{cells} and the efficiency of the column drilling process is above \SI{99}{\%}. The first prototypes of small cell 3D diamond pixel detectors read out more charge than any planar \ac{pCVD} diamond detector. The measured relative hit efficiency of the 3D pixel detectors reached \SI{99.3}{\%} compared to a planar silicon device.\par
It was found that irradiated \ac{pCVD} diamond detectors work reliably and there is no signal variation greater than \SI{2}{\%} up to an incident particle flux of \SI{20}{\mhzcm}. This was shown for a range of irradiations up to a maximum fluence of \SI{8e15}{\ncm}. The beam induced current of a \ac{pCVD} diamond is proportional to the flux and the leakage current is of the \orderof{\SI{1}{\nano\ampere}}. It was also demonstrated that it is possible to correct a large rate dependence that occurs in a small fraction of diamonds and is most likely due to surface properties.
