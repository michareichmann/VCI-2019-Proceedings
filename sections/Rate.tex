\section{High Rate Studies}
% Since the detectors have to operate in a very harsh radiation environment 
Since the \ac{HL-LHC} will reach particle fluxes of \orderof{\SI{}{\giga\hertz\per\centi\meter^2}} it is very important to understand the effect of the incident particle flux on the signal of \ac{pCVD} diamonds.
%%%%%%%%%%%%%%%%%%%%%%% TEST SITE %%%%%%%%%%%%%%%%%%%%%%%%%%%
\paragraph{Test Site}
In order to conduct such a study it is essential to be able to vary the particle flux over a large range. The $\uppi$M1 beam line at the \ac{HIPA} at \ac{PSI} \cite{hipa} can provide beams with continuously tunable fluxes from the order of \SI{1}{\khzcm} up to \SI{20}{\mhzcm}. It is a bunched beam with a spacing of \SI{19.7}{\nano\second}. For these studies a $\uppi^{\z{+}}$ beam with with a momentum of \SI{260}{\mega\electronvolt\per c}  was chosen in order to reach the highest possible flux \cite{pim1}.\par
%%%%%%%%%%%%%%%%%%%%%%% SETUP %%%%%%%%%%%%%%%%%%%%%%%%%%%
\paragraph{Setup}
The diamond sensors were connected in a pad geometry and prepared as described in \cite{rainer}. 
In order to resolve single waveforms at high particle rates the sensors were connected to a fast, low-noise amplifier with a rise time of approximately \SI{5}{\nano\second}. The resulting waveforms are then digitised and recorded in a beam telescope setup which provides spatial information of the hits in the diamond detector.\par
%%%%%%%%%%%%%%%%%%%%%%% RESULTS %%%%%%%%%%%%%%%%%%%%%%%%%%%
\paragraph{Results}
\ac{pCVD} diamond has a interior crystal structure where the single grains have slightly different properties. Due to that the size of the measured signal in \ac{pCVD} depends also on the spatial position as can be seen in \ar{sm}. However this behaviour is constant and does not depend on time or rate.\par
As an important factor also the leakage current is studied depending on the particle rate. In \SI{80}{\%} of the measured diamonds the current is proportional and the base current without beam is of the \orderof{\SI{1}{\nano\ampere}}. Some bad diamonds show shifting base lines or even erratic currents \cite{erratic}.\par
In order to measure the signal behaviour depending on the incident particle 
\wrapfig[r]{.35}{SMH8}[Signal map of a \ac{pCVD} detector.][sm]
flux several rate scans with both polarities of the bias voltage and different irradiation doses were performed. \ar{rate3} shows the final results for a \ac{pCVD} diamond with various irradiations with reactor neutrons up to \SI{8e15}{\ncm}. An upper limit of the pulse height dependence on particle flux for irradiated devices of less than \SI{2}{\%} was observed for a flux up to \SI{20}{\mega\hertz\per cm^2}. But we also observed a single diamond that showed a very strong rate dependence of more than \SI{90}{\%}. However, after the sensor was cleaned with \ac{RIE} and fully reprocessed a new measurement showed a dependence of less than \SI{2}{\%}. This lead us to the conclusion that these effects result from surface contamination and it is possible to repair them.\par
\fig[r]{.3}{B2Scans2}[Pulse height versus incident particle flux for a \ac{pCVD} diamond for different irradiations][rate3]
After the irradiation the pulse height decreases due to the radiation damage. There was no absolute calibration done yet which is required to relate the pulse height values before and after irradiation.\par
