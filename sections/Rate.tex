\section{High Rate Studies}
At the \ac{HL-LHC} particle fluxes will reach the \orderof{\SI{}{\giga\hertz\per\centi\meter^2}} hence it is very important to understand the effect of the incident particle flux on the signal of all prospective detectors. In order to conduct a high rate study it is necessary to be able to vary the particle flux over a large range. The $\uppi$M1 beam line at the \ac{HIPA} at \ac{PSI} \cite{hipa} can provide beams with continuously tunable fluxes from the order of \SI{1}{\khzcm} up to \SI{20}{\mhzcm}. The $\uppi$M1 beam is bunched with a spacing of \SI{19.7}{\nano\second}. For these studies a $\uppi^{\z{+}}$ beam with a momentum of \SI{260}{\mega\electronvolt\per c}  was chosen in order to reach the highest possible flux \cite{pim1}. In total 13 \ac{pCVD} diamonds were measured which were all prepared in the same way.\par
%%%%%%%%%%%%%%%%%%%%%%% SETUP %%%%%%%%%%%%%%%%%%%%%%%%%%%
\subsection{Setup}
The planar diamond sensors were connected in a pad geometry and prepared as described in \cite{felix}. 
In order to resolve individual particles at high particle rates the sensors were connected to a fast amplifier with low electronic noise and a rise time of approximately \SI{5}{\nano\second}. The resulting waveforms were digitised and recorded in a beam telescope setup \cite{felix} which provides spatial information of the hits in the diamond detector. Due to the low momentum of the incident particles the spatial resolution of the telescope was of the \orderof{\SI{100}{\micro\meter}}.\par
%%%%%%%%%%%%%%%%%%%%%%% RESULTS %%%%%%%%%%%%%%%%%%%%%%%%%%%
\subsection{Results}
In order to measure the signal behaviour as a function of incident particle flux and irradiation, several rate scans with both 
\fig{.29}{B2Scans}[Pulse height versus incident particle flux for a \ac{pCVD} diamond for various fluences at a bias voltage of \SI{-1000}{\volt}.][rate3]
polarities of the bias voltage were performed. \ar{rate3} shows the preliminary results for a \ac{pCVD} diamond with 
various fluences up to a maximum particle flux of \SI{20}{\mhzcm} at a bias voltage of \SI{-1000}{\volt}. The sensor was irradiated with fast reactor neutrons in steps up to total fluence of \SI{8e15}{\ncm} at the irradiation facilities at the JSI TRIGA reactor in Ljubljana \cite{irrad}. The mean pulse height of the single rate scans is scaled to 1. The results show that the pulse height is flat with respect to the flux deviating less than \SI{2}{\%} from the mean.\par 
%%%%%%%%%%%%%%%%%%%%%%%%%%%%%%%%%%%%%%%%%%%%%%%%%%%%%%%%%%
The effect of particle rate on the beam induced current in diamond detectors was also measured. \SI{80}{\%} of the measured diamonds had currents proportional to the flux and a leakage current without a beam of the \orderof{\SI{1}{\nano\ampere}}. The other \SI{20}{\%} show shifting base lines or erratic dark currents \cite{erratic}. These diamonds are considered problematic and were not analysed for this article.\par
%%%%%%%%%%%%%%%%%%%%%%%%%%%%%%%%%%%%%%%%%%%%%%%%%%%%%%%%%%
\ac{pCVD} diamond has an interior crystal structure where the individual grains have slightly different properties. Therefore the size of the measured signal in \ac{pCVD} diamond can also depend on the spatial position as can be seen in \ar{sm}. A constant fiducial region was used to minimise any effects of the spatial dependence. \par % However this behaviour is constant and neither depends on time nor on rate.\par
\fig{.25}{SM97}[Pulse height map of a \ac{pCVD} diamond as a function of spatial position.][sm]
%%%%%%%%%%%%%%%%%%%%%%%%%%%%%%%%%%%%%%%%%%%%%%%%%%%%%%%%%%
We also observed a single diamond with a large rate dependence losing \SI{90}{\%} of the signal at the highest rate. After the surface was cleaned and processed with \ac{RIE}, the device was re-metallised. A new measurement showed a deviation of less than \SI{2}{\%} from the mean pulse height. This leads us to the conclusion that this rate effect was due to surface properties and is possible to repair.
% After the irradiation the pulse height decreases due to the radiation damage. There was no absolute calibration done yet which is required to relate the pulse height values before and after irradiation.\par
