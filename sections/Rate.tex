\section{High Rate Studies}
The \ac{HL-LHC} will reach particle fluxes of \orderof{\SI{}{\giga\hertz\per\centi\meter^2}} hence it is very important to understand the effect of the incident particle flux on the signal of \ac{pCVD} diamonds. In order to conduct such a study it is essential to be able to vary the particle flux over a large range. The $\uppi$M1 beam line at the \ac{HIPA} at \ac{PSI} \cite{hipa} can provide beams with continuously tunable fluxes from the order of \SI{1}{\khzcm} up to \SI{20}{\mhzcm}. The $\uppi$M1 beam is a bunched beam with a spacing of \SI{19.7}{\nano\second}. For these studies a $\uppi^{\z{+}}$ beam with with a momentum of \SI{260}{\mega\electronvolt\per c}  was chosen in order to reach the highest possible flux \cite{pim1}.\par
%%%%%%%%%%%%%%%%%%%%%%% SETUP %%%%%%%%%%%%%%%%%%%%%%%%%%%
\subsection{Setup}
The diamond sensors were connected in a pad geometry and prepared as described in \cite{rainer}. 
In order to resolve single waveforms at high particle rates the sensors were connected to a fast, low-noise amplifier with a rise time of approximately \SI{5}{\nano\second}. The resulting waveforms were digitised and recorded in a beam telescope setup which provides spatial information of the hits in the diamond detector.\par
%%%%%%%%%%%%%%%%%%%%%%% RESULTS %%%%%%%%%%%%%%%%%%%%%%%%%%%
\subsection{Results}
\ac{pCVD} diamond has a interior crystal structure where the single grains have slightly different properties. Therefore the size of the measured signal in \ac{pCVD} depends also on the spatial position as can be seen in \ar{sm}. However this behaviour is constant and does not depend on time or rate.\par
\fig{.25}{SM97}[Signal map of a \ac{pCVD} detector.][sm]
As an important factor also the beam induced current is studied depending on the particle rate. In \SI{80}{\%} of the measured diamonds the current is proportional to the flux and the leakage current (no beam) is of the \orderof{\SI{1}{\nano\ampere}}. Some bad diamonds show shifting base lines or even erratic currents \cite{erratic}. These are not shown in this article.\par
In order to measure the signal behaviour as a function of incident particle flux and irradiation, several rate scans with both polarities of the bias voltage were performed. \ar{rate3} shows the final results for a \ac{pCVD} diamond with 
\fig[r]{.29}{B2Scans}[Pulse height versus incident particle flux for a \ac{pCVD} diamond for various fluences at \SI{-1000}{\volt}.][rate3]
various fluences. The sensors were irradiated at the irradiation facilities at the JSI TRIGA reactor in Ljubljana with fast reactor neutrons up to \SI{8e15}{\ncm} \cite{irrad}. The mean of the single scans is scaled to 1. The pulse height is flat with the flux deviating less than \SI{2}{\%} from the mean up to a flux of \SI{20}{\mhzcm}. We also observed a single diamond with a large rate dependence losing \SI{90}{\%} of the signal at highest rate. After the surface was processed with cleaning and \ac{RIE}, the device was reprocessed. A new measurement showed a deviation of less than \SI{2}{\%}. This leads us to the conclusion that this rate effect is due to surface issues and is possible to repair.
% After the irradiation the pulse height decreases due to the radiation damage. There was no absolute calibration done yet which is required to relate the pulse height values before and after irradiation.\par
